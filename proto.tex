\chapter{オブジェクト}
\label{chap:proto}

この章ではECMAScript5.1(以下ES5)をベースにしてオブジェクトについて
学びます。
流れとしてはES5仕様書のオブジェクトの章と
それに関連する項目についての意訳と補足といった感じで進めていきます。

\section{プリミティブ型とオブジェクト型}

ECMAScript値の型として6つの型があります。
プリミティブ型として
Undefined(\jsundef)、
Boolean({\jstrue} 、\jsfalse)、
Null(\jsnull)、
Number(数値)、
String(文字列)の5種類が存在し、
それ以外のものはObject(オブジェクト型)となります。

\section{オブジェクトとは}

オブジェクト(オブジェクト型のデータ)は一言で言うとプロパティの集合です。
プロパティにはコード上でプロパティ名を指定してアクセスできるプロパティである
データプロパティとアクセサプロパティ、内部的に使われる
内部プロパティ\footnote{[[Hoge]]のように記述します}の3種類に分類されます。

\section{プロパティへのアクセス}
\label{sec:prop_access}

いわゆるドット記法かハッシュ記法でプロパティ名を指定してアクセスします。

\begin{lstlisting}
var a = { foo: 1 };
// ドット記法
a.foo;
// ハッシュ記法
a['foo'];
\end{lstlisting}


\section{メソッド}

オブジェクトのプロパティの中でも特に関数オブジェクトであるものをメソッドと呼びます。
オブジェクトOのプロパティ名Fのメソッドを呼び出したときに
\jsthis をOとしてメソッド内の内部メソッド\jscall が呼ばれます。

\begin{lstlisting}
var a = {
  hoge: 1,
  foo: function() { console.log(this.hoge) }
};
a.foo(); // 1

b.hoge = 2;
b.foo = a.foo;
b.foo(); // 2
\end{lstlisting}

\section{プロパティ属性}
\label{sec:prop_attr}

オブジェクトのプロパティは単純なハッシュというわけでなくて、
内部的に属性を持っており、プロパティにアクセスしたときの挙動は
この属性によって決まります。
データプロパティは\jsvalue 、\jswritable 、\jsenumerable 、\jsconfigurable
、アクセサプロパティは\jsget 、\jsset 、\jsenumerable 、\jsconfigurable
属性を持ちます。

\subsection{\jsvalue}

プロパティが持つ実際の値です。
ES5で定義されている任意の値をとります。
デフォルト値は\jsundef です。

\subsection{\jswritable}

プロパティの\jsvalue 属性が書き換え可能かどうかを表します。
Booleanの値をとります。
{\jsfalse}の場合、プロパティは読み出し専用となります。
デフォルト値は{\jsfalse}です。

\subsection{\jsget}

プロパティを読み出すときに呼び出される、いわゆるゲッターです。
関数オブジェクトか\jsundef の値をとります。
\jsget が関数オブジェクトの場合、
引数なしで呼び出され、その結果がプロパティの値として返ります。
デフォルト値は\jsundef です。

\subsection{\jsset}

プロパティを書き込むときに呼び出される、いわゆるセッターです。
関数オブジェクトか\jsundef の値をとります。
\jsset が関数オブジェクトの場合、1つの引数で呼び出されます。
デフォルト値は\jsundef です。

\subsection{\jsenumerable}

Booleanの値をとります。\jstrue の場合、for-in文でプロパティ名が列挙されます。
デフォルト値は{\jsfalse}です。

\subsection{\jsconfigurable}

Booleanの値をとります。{\jsfalse}の場合、プロパティの属性を変更(ただし、\jsvalue の値は変更できる)
することやプロパティの削除ができなくなります。
デフォルト値は{\jsfalse}です。

\section{プロパティディスクリプタ}

プロパティディスクリプタは
プロパティの属性をオブジェクトリテラル風に表記したものです。
これを使うとプロパティが持っている属性自体や、プロパティの属性を操作するときに
わかりやすく表現できます。
(例 : \jsvalue が\jstrue であるプロパティディスクリプタは\{\jsvalue : {\jstrue}\}のように
表記します)。

プロパティディスクリプタには3種類あって、データプロパティの属性を表現する
データディスクリプタ
(\jsvalue 、 \jswritable の内少なくとも1つは存在する)、
アクセサプロパティの属性を表現するアクセサディスクリプタ
(\jsget 、\jsset の内少なくとも1つは存在する)、
そのどちらでもないものをジェネリックディスクリプタと呼びます。
ただし、データディスクリプタでありアクセサディスクリプタでもあるようなプロパティディスクリプタは
ありません。

プロパティディスクリプタはあくまでも内部的なものですが、
ES5では実際のコード上で
プロパティディスクリプタをオブジェクト形式で参照したり、
プロパティディスクリプタ風の記述方式で
プロパティの属性を変更したり、プロパティ自体を定義することができるようになりました。

プロパティディスクリプタの種類を判定したり、オブジェクトを
プロパティディスクリプタに変換するアルゴリズムについては以下で解説します。

\subsection{IsAccessorDescriptor(Desc)}

プロパティディスクリプタDescがアクセサディスクリプタかどうか判定します。

\begin{enumerate}
 \item Descが\jsundef な場合、{\jsfalse}を返します
 \item Desc.\jsget とDesc.\jsset が両方とも存在しない場合、{\jsfalse}を返します
 \item {\jstrue}を返します
\end{enumerate}

\subsection{IsDataDescriptor(Desc)}

プロパティディスクリプタDescが
データディスクリプタかどうか判定します。

\begin{enumerate}
 \item Descが\jsundef な場合、{\jsfalse}を返します
 \item Desc.\jsvalue とDesc.\jswritable が両方とも存在しない場合、{\jsfalse}を返します
 \item {\jstrue}を返します
\end{enumerate}

\subsection{IsGenericDescriptor(Desc)}

プロパティディスクリプタDescが
ジェネリックディスクリプタかどうか判定します。

\begin{enumerate}
 \item Descが\jsundef な場合、{\jsfalse}を返します
 \item IsAccessorDescriptor(Desc)とIsDataDescriptor(Desc)が
 両方とも{\jsfalse}であるとき{\jstrue}を返します
 \item {\jsfalse}を返します
\end{enumerate}

\subsection{FromPropertyDescriptor(Desc)}

プロパティディスクリプタDescをECMAScriptのオブジェクト
で表現したものに変換します。

\begin{enumerate}
 \item Descが\jsundef な場合、\jsundef を返します
 \item objをnew Object()した結果とします
 \item IsDataDescriptor(Desc)が{\jstrue}な場合、
 \begin{enumerate}
  \item objの内部メソッド\jsdefownprop を引数\texttt{'value'}、
  プロパティディスクリプタ
  \{
  \jsvalue : Desc.\jsvalue ,
  \jswritable : {\jstrue} ,
  \jsenumerable : {\jstrue} ,
  \jsconfigurable : {\jstrue}
  \}、{\jsfalse}として呼び出します
  \item objの内部メソッド\jsdefownprop を引数\texttt{'writable'}、
  プロパティディスクリプタ
  \{
  \jsvalue : Desc.\jswritable ,
  \jswritable :{\jstrue} ,
  \jsenumerable : {\jstrue} ,
  \jsconfigurable : {\jstrue}
  \}、{\jsfalse}として呼び出します
 \end{enumerate}
 \begin{enumerate}
  \item objの内部メソッド\jsdefownprop を引数\texttt{'get'}、
  プロパティディスクリプタ
  \{
  \jsvalue : Desc.\jsget ,
  \jswritable : {\jstrue} ,
  \jsenumerable : {\jstrue} ,
  \jsconfigurable : {\jstrue}
  \}、{\jsfalse}として呼び出します
  \item objの内部メソッド\jsdefownprop を引数\texttt{'set'}、
  プロパティディスクリプタ
  \{
  \jsvalue : Desc.\jsset ,
  \jswritable :{\jstrue} ,
  \jsenumerable : {\jstrue} ,
  \jsconfigurable : {\jstrue}
  \}、{\jsfalse}として呼び出します
 \end{enumerate}
 \item objの内部メソッド\jsdefownprop を引数\texttt{'enumerable'}、
 プロパティディスクリプタ
 \{
 \jsvalue : Desc.\jsenumerable ,
 \jswritable :{\jstrue} ,
 \jsenumerable : {\jstrue} ,
 \jsconfigurable : {\jstrue}
 \}、{\jsfalse}として呼び出します
 \item objの内部メソッド\jsdefownprop を引数\texttt{'configurable'}、
 プロパティディスクリプタ
 \{
 \jsvalue : Desc.\jsconfigurable ,
 \jswritable :{\jstrue} ,
 \jsenumerable : {\jstrue} ,
 \jsconfigurable : {\jstrue}
 \}、{\jsfalse}として呼び出します
 \item objを返します
\end{enumerate}

\subsection{ToPropertyDescriptro(Obj)}

オブジェクトObjをプロパティディスクリプタに変換します。

\begin{enumerate}
 \item Oがオブジェクト型でない場合TypeError例外を投げます
 \item フィールドを持たないプロパティディスクリプタdescを作ります
 \item Objの内部メソッド\jshasprop を引数を\texttt{'enumerable'}として呼び出した結果が
 {\jstrue} な場合、
 \begin{enumerate}
  \item enumをObjの内部メソッド\jsget を引数を\texttt{'enumerable'}として呼び出した結果とします
  \item desc.\jsenumerable にToBoolean(enum)\footnote{ToBooleanは値をBoolean値に変換する抽象的な処理を表します。}をセットします
 \end{enumerate}
 \item Objの内部メソッド\jshasprop を引数を\texttt{'configurable'}として呼び出した結果が
 {\jstrue} な場合、
 \begin{enumerate}
  \item confをObjの内部メソッド\jsget を引数を\texttt{'configurable'}として呼び出した結果とします
  \item desc.\jsconfigurable にToBoolean(conf)をセットします
 \end{enumerate}
 \item Objの内部メソッド\jshasprop を引数を\texttt{'configurable'}として呼び出した結果が
 {\jstrue} な場合、
 \begin{enumerate}
  \item valueをObjの内部メソッド\jsget を引数を\texttt{'value'}として呼び出した結果とします
  \item desc.\jsvalue にToBoolean(value)をセットします
 \end{enumerate}
 \item Objの内部メソッド\jshasprop を引数を\texttt{'writable'}として呼び出した結果が
 {\jstrue} な場合、
 \begin{enumerate}
  \item writableをObjの内部メソッド\jsget を引数を\texttt{'writable'}として呼び出した結果とします
  \item desc.\jswritable にToBoolean(writable)をセットします
 \end{enumerate}
 \item Objの内部メソッド\jshasprop を引数を\texttt{'get'}として呼び出した結果が
 {\jstrue} な場合、
 \begin{enumerate}
  \item getterをObjの内部メソッド\jsget を引数を\texttt{'get'}として呼び出した結果とします
  \item IsCallable(getter)
  \footnote{IsCallableは内部メソッド\jscall を持つオブジェクトかどうか(つまり、関数オブジェクト)判定します}
  が{\jsfalse}でgetterが\jsundef でない場合、
  TypeError例外を投げます
  \item desc.\jsget にgetterをセットします
 \end{enumerate}
 \item Objの内部メソッド\jshasprop を引数を\texttt{'set'}として呼び出した結果が
 {\jstrue} な場合、
 \begin{enumerate}
  \item setterをObjの内部メソッド\jsget を引数を\texttt{'set'}として呼び出した結果とします
  \item IsCallable(setter)が{\jsfalse}でsetterが\jsundef でない場合、
  TypeError例外を投げます
  \item desc.\jsset にsetterをセットします
 \end{enumerate}
 \item desc.\jsget かdesc.\jsset が存在したときに、
 \begin{enumerate}
  \item desc.\jsvalue かdesc.\jswritable が存在する場合、
  TypeError例外を投げます
 \end{enumerate}
 \item descを返します
\end{enumerate}

\section{内部プロパティ・メソッド}

オブジェクトは内部的なプロパティやメソッドを持ちます。
コード上からプロパティを読み出し、書き込み、削除したりするときには
内部でこれらのプロパティ・メソッドが使われています。

\subsection{\jsproto}

いわゆるオブジェクトのプロトタイプというやつです。
内部プロパティ\jsproto は任意のオブジェクト型か\jsnull の値を持ちます。
つまり\jsproto がオブジェクト型である場合は、それ自体も
内部プロパティ\jsproto を持ちます。このような\jsproto が連鎖するような構造を
プロトタイプチェーンと呼びます。

\subsubsection*{Object.prototype.\_\_proto\_\_}

ES5標準ではありませんが、事実上\texttt{\_\_proto\_\_}プロパティがオブジェクトの
内部プロパティ\jsproto を操作するためのアクセサプロパティとなっています。
このプロパティは\texttt{Object.prototype}に定義されていて、
それを継承したオブジェクトから使うことができます。ただし、実装されていない
環境もあるので注意が必要です。

\begin{lstlisting}
var o = {};
var proto = { foo: 1 };

// [[Prototype]]にオブジェクトを書き込みます
o.__proto__ = proto;
console.log(o.foo); // 1

// [[Prototype]]を読み出します
console.log(o.__proto__ === proto); // true
\end{lstlisting}

\subsection{\jsclass}

オブジェクトを分類するためにつけられる文字列です。
これは\texttt{Object.prototype.toString}メソッドから取得することができます。
独自の文字列を設定する方法はありません。

\begin{lstlisting}
var toString = Object.prototype.toString;
// \jsclass は"[object Hoge]"の"Hoge"の部分です
console.log(toString.call([])); // "[object Array]"
console.log(toString.call(function() {})); // "[object Function]"
\end{lstlisting}

\subsection{\jsextensible}

オブジェクトが拡張可能かどうか判定するためのフラグです。
Booleanの値をとります。

\subsection{\jsget(P)}

オブジェクトOのプロパティ名Pのプロパティを読みだすときに
呼ばれる内部メソッドです。

\begin{enumerate}
 \item descをOの内部メソッド\jsgetprop をプロパティ名Pとして呼び出した返り値とします
 \item もしdescが\jsundef の場合、\jsundef を返します
 \item IsDataDescriptor(desc)が {\jstrue}  である場合、desc.\jsvalue を返します
 \item それ以外の場合はIsAccessorDescriptor(desc)なはずので、getterをdesc.\jsget とします。
 \item getterが\jsundef の場合、\jsundef を返します
 \item getterの内部メソッド\jscall を\jsthis をO、引数なしで呼び出した結果を返します
\end{enumerate}


\subsection{\jsgetownprop(P)}

オブジェクトO自身が持つプロパティ名Pのプロパティディスクリプタを返します。
返り値は内部的に使われます。

\begin{enumerate}
 \item Oが自分自身のプロパティ名Pというプロパティを
 持たない場合、\jsundef を返します
 \item Dを属性を持たない新しいプロパティディスクリプタとします
 \item XをOのプロパティ名Pのプロパティとします
 \item Xがデータプロパティの場合
 \begin{enumerate}
  \item D.\jsvalue にXの\jsvalue 属性をセットします
  \item D.\jswritable にXの\jswritable 属性をセットします
 \end{enumerate}
 \item Xがアクセサプロパティの場合
 \begin{enumerate}
  \item D.\jsget にXの\jsget 属性をセットします
  \item D.\jsset にXの\jsset 属性をセットします
 \end{enumerate}
 \item D.\jsenumerable にXの\jsenumerable 属性をセットします
 \item D.\jsconfigurable にXの\jsconfigurable 属性をセットします
 \item Dを返します
\end{enumerate}

String型での\jsgetownprop はここで説明したものと異なります(ここでは割愛します)。

\subsection{\jsgetprop(P)}

オブジェクトOが持つプロパティ名Pのプロパティディスクリプタを返します。
返り値は内部的に使われます。

\begin{enumerate}
 \item propをOの内部メソッド\jsgetownprop をプロパティ名Pとして呼び出した返り値とします
 \item もしpropが\jsundef でない場合、propを返します
 \item protoをOの内部プロパティ\jsproto とします
 \item もしprotoが\jsnull の場合、\jsundef を返します
 \item protoの内部メソッド\jsgetprop をプロパティ名Pで呼び出した結果を返します
\end{enumerate}

\subsection{\jscanput(P)}

オブジェクトOのプロパティ名Pのプロパティに値を代入できるかをBoolean値で返します。
返り値は内部的に使われます。

\begin{enumerate}
 \item descをOの内部メソッド\jsgetownprop を引数Pで呼び出した結果とします
 \item descが\jsundef でない場合、
 \begin{enumerate}
  \item IsAccessorDescriptor(desc)が{\jstrue} である場合、
  \begin{enumerate}
   \item desc.\jsset が\jsundef な場合、{\jsfalse}を返します
   \item 違う場合は{\jstrue} を返します
  \end{enumerate}
  \item 違う場合は、データディスクリプタになるのでdesc.\jswritable を返します
 \end{enumerate}
 \item protoをOの内部プロパティ\jsproto とします
 \item もしprotoが\jsnull な場合、Oの内部プロパティ\jsextensible を返します
 \item inheritedをprotoの内部メソッド\jsgetprop をプロパティ名Pで呼び出した結果とします
 \item inheritedが\jsundef な場合、Oの内部プロパティ\jsextensible を返します
 \item IsAccessorDescriptor(inherited)が{\jstrue} な場合、
 \begin{enumerate}
  \item inherited.\jsset が\jsundef な場合、{\jsfalse}を返します
  \item 違う場合、{\jstrue} を返します
 \end{enumerate}
 \item 違う場合、inheritedは必ずデータディスクリプタになります
 \begin{enumerate}
  \item Oの内部プロパティ\jsextensible が{\jsfalse}な場合、{\jsfalse}を返します
  \item 違う場合、inherited.\jswritable を返します
 \end{enumerate}
\end{enumerate}

\subsection{\jsput(P, V, Throw)}

オブジェクトOのプロパティ名Pのプロパティに値Vを書き込もうとしたときに
呼ばれる内部メソッドです。

\begin{enumerate}
 \item Oの内部メソッド\jscanput をプロパティ名Pとして呼び出した返り値が{\jsfalse}の場合、
 \begin{enumerate}
  \item Throwが{\jstrue} な場合、TypeError例外を投げます
  \item 違う場合は終了します
 \end{enumerate}
 \item ownDescをOの内部メソッド\jsgetownprop をプロパティ名Pとして呼び出した返り値とします
 \item もしIsDataDescriptor(ownDesc)が{\jstrue} である場合、
 \begin{enumerate}
  \item valueDescをプロパティディスクリプタ\{\jsvalue : V\}とします。
  \item Oの内部メソッド\jsdefownprop を引数をP、valueDesc、Throwとして呼び出します
  \item 終了します
 \end{enumerate}
 \item descをOの内部メソッド\jsgetprop を引数をPとして呼び出した返り値とします。
 これは自分自身か継承されたアクセサプロパティ、もしくは継承されたデータプロパティです。
 \item IsAccessorDescriptor(desc)が{\jstrue} である場合、
 \begin{enumerate}
  \item setterをdesc.\jsset とします(ただし、\jsundef でない限り)
  \item setterの内部メソッド\jscall をthisをO、引数をVとして呼び出します
 \end{enumerate}
 \item 違う場合、Oにプロパティ名Pのデータプロパティを作ります
 \begin{enumerate}
  \item newDescをプロパティディスクリプタ\{\jsvalue : V, \jswritable : {\jstrue} , \jsenumerable : true, jsconfigurable : {\jstrue} \}
  とします
  \item Oの内部メソッド\jsdefownprop を引数をP、newDesc、Throwとして呼び出します
 \end{enumerate}
 \item 終了します
\end{enumerate}

\subsubsection*{継承されたプロパティがある場合に値を書き込むときのややこしい挙動}

オブジェクト自身にはプロパティが存在しないが
継承されたプロパティがあるときに
プロパティを書き込むときにプロパティが
新しく定義されないパターンが3種類あります。

\begin{enumerate}
 \item 継承されたプロパティがデータプロパティで
 \jswritable 属性が\jsfalse
 \item 継承されたプロパティがアクセサプロパティで
 \jsset が関数オブジェクト
 \item 継承されたプロパティがアクセサプロパティで
 \jsset が\jsundef
\end{enumerate}

\begin{lstlisting}
// データプロパティで[[Writable]]属性がfalseの場合は何もしない
var proto = Object.create(null, {
  foo: { value: 1 }
});
var o = Object.create(proto);
o.foo = 2;
console.log(o.foo); // 1;

// アクセサプロパティで[[Set]]属性が関数オブジェクトな場合はそれが呼ばれる
var proto = Object.create(null, {
  foo: { set: function(v) { this._foo = v } }
});
var o = Object.create(proto);
o.foo = 2;
console.log(o.foo); // undefined
console.log(o._foo); // 2

// アクセサプロパティで[[Set]]属性がundefinedな場合はなにもしない
var proto = Object.create(null, {
  foo: { get: function() { return 1 } }
});
var o = Object.create(proto);
o.foo = 2;
console.log(o.foo); // 1
\end{lstlisting}

\subsection{\jshasprop(P)}

オブジェクトO自身がプロパティ名Pのプロパティを持っているかをBoolean値で返します。
返り値は内部的に使われます。

\begin{enumerate}
 \item descをOの内部メソッド\jsgetprop をプロパティ名Pとして呼び出した結果とします
 \item descが\jsundef な場合、{\jsfalse}を返します
 \item ちがう場合、{\jstrue} を返します
\end{enumerate}

\subsection{\jsdelete(P, Throw)}

オブジェクトO自身のプロパティ名Pのプロパティを削除するときに
呼ばれる内部メソッドです。
削除できたかどうかをBoolean値で返します。
Throwはstrict codeであるときに{\jstrue}となります。

\begin{enumerate}
 \item descをOの内部メソッド\jsgetownprop をプロパティ名Pとして呼び出した結果とします
 \item descが\jsundef な場合、{\jstrue} を返します
 \item desc.\jsconfigurable が{\jstrue} な場合、
 \begin{enumerate}
  \item O自身が持つプロパティ名Pのプロパティを削除します
  \item {\jstrue} を返します
 \end{enumerate}
 \item Throwが{\jstrue} な場合、TypeError例外を投げます
 \item {\jsfalse}を返します
\end{enumerate}

\subsection{\jsdefault(hint)}

与えられたヒント(String, Number)より
オブジェクトをプリミティブ値に変換します\footnote{例えば、暗黙的な型変換が行われるときに呼ばれます}。
\jsdefault は以下の手順で実行されます。

ヒントがStringである場合、

\begin{enumerate}
 \item toStringをOの内部メソッド\jsget を引数を'toString'として呼び出した結果とします
 \item IsCallable(toString)が{\jstrue} な場合、
 \begin{enumerate}
  \item strはtoStringの内部メソッド\jscall を\jsthis をO、引数なしで呼び出した結果とします
  \item strがプリミティブ値な場合、strを返します
 \end{enumerate}
 \item valueOfをOの内部メソッド\jsget を引数を'valueOf'として呼び出した結果とします
 \item IsCallable(valueOf)が{\jstrue} な場合、
 \begin{enumerate}
  \item valを内部メソッド\jscall を\jsthis をO、引数なしで呼び出した結果とします
  \item valがプリミティブ値な場合、valを返します
 \end{enumerate}
 \item TypeError例外を投げます
\end{enumerate}

ヒントがNumberである場合、

\begin{enumerate}
 \item valueOfをOの内部メソッド\jsget を引数を'valueOf'として呼び出した結果とします
 \item IsCallable(valueOf)が{\jstrue} な場合、
 \begin{enumerate}
  \item valを内部メソッド\jscall を\jsthis をO、引数なしで呼び出した結果とします
  \item valがプリミティブ値な場合、valを返します
 \end{enumerate}
 \item toStringをOの内部メソッド\jsget を引数を'toString'として呼び出した結果とします
 \item IsCallable(toString)が{\jstrue} な場合、
 \begin{enumerate}
  \item strはtoStringの内部メソッド\jscall を\jsthis をO、引数なしで呼び出した結果とします
  \item strがプリミティブ値な場合、strを返します
 \end{enumerate}
 \item TypeError例外を投げます
\end{enumerate}

\begin{lstlisting}
var o = { foo: 1, bar: 2 };

// Object.prototype.toStringが呼ばれる
console.log('' + o); // [object Object]

// Object.prototype.valueOfが呼ばれる
console.log(+o); // NaN

Object.defineProperty(o, 'toString', {
  value: function() {
    return Object.keys(this).map(function(k) {
      return k + ':' + this[k];
    }.bind(this)).join();
  },
  writable: true,
  configurable: true
});

// 定義したtoStringが呼ばれる
console.log('' + o); // [object Object]

Object.defineProperty(o, 'valueOf', {
  value: function() {
    return 1;
  },
  writable: true,
  configurable: true
});

// 定義したvalueOfが呼ばれる
console.log(+o); // 1
\end{lstlisting}


\subsection{\jsdefownprop(P, Desc, Throw)}

オブジェクトO自身にプロパティ名Pのプロパティを定義するときに呼ばれます。
プロパティはプロパティディスクリプタDescを使って定義されます。
以下の実行手順で使われる'Reject'は'Throwが{\jstrue} な場合、TypeError例外を投げ、それ以外は{\jsfalse}を返す'
という意味です。

\begin{enumerate}
 \item currentをOの内部メソッド\jsgetownprop をプロパティ名Pとして呼び出した結果とします
 \item extensibleをOの内部プロパティ\jsextensible とします
 \item currentが\jsundef かつextensibleが{\jsfalse}な場合、Rejectします
 \item currentが\jsundef かつextensibleが{\jstrue} な場合、
 \begin{enumerate}
  \item IsGenericDescriptor(Desc)かIsDataDescriptor(Desc)が{\jstrue} な場合、
  \begin{enumerate}
   \item O自身にプロパティ名Pのデータプロパティを作ります。
   \jsvalue 、\jswritable 、\jsenumerable 、\jsconfigurable 属性の値は
   Descと対応するものになります。ただし、フィールドがない属性については、
   属性のデフォルト値が割り当てられます
  \end{enumerate}
  \item 違う場合、Descはアクセサディスクリプタとなります
  \begin{enumerate}
   \item O自身にプロパティ名Pのアクセサプロパティを作ります。
   \jsget 、\jsset 、\jsenumerable 、\jsconfigurable 属性の値は、
   Descと対応するものになります。ただし、フィールドがない属性については、
   属性のデフォルト値が割り当てられます
  \end{enumerate}
  \item {\jstrue} を返します
 \end{enumerate}
 \item Descの全てのフィールドがない場合、{\jstrue} を返します
 \item Descの全てのフィールドがcurrentと同じ場合、{\jstrue} を返します
 \item current.\jsconfigurable が{\jsfalse}な場合、
 \begin{enumerate}
  \item Desc.\jsconfigurable が{\jstrue} な場合、Rejectします
  \item Desc.\jsenumerable が存在しSameValue(Desc.\jsenumerable , current.\jsenumerable) が{\jsfalse}な場合、Rejectします
 \end{enumerate}
 \item IsGenericDescriptor(Desc)が{\jstrue} な場合、バリデーションは不要です
 \item そうでなくて、IsDataDescriptor(current)とIsDataDescriptor(Desc)が
 違う結果の場合、
 \begin{enumerate}
  \item current.\jsconfigurable が{\jsfalse}な場合、Rejectします
  \item IsDataDescriptor(current)が{\jstrue} の場合、Oのプロパティ名Pのプロパティをデータプロパティから
  アクセサプロパティに変換します。
  \jsconfigurable 、\jsenumerable 属性はそのままにして、他の属性はそれぞれのデフォルト値とします。
  \item 違う場合、Oのプロパティ名Pのプロパティをアクセサプロパティから
  データプロパティに変換します。
  \jsconfigurable 、\jsenumerable 属性はそのままにして、他の属性はそれぞれのデフォルト値とします。
 \end{enumerate}
 \item そうでなくて、IsDataDescriptor(current)とIsDataDescriptor(Desc)
 双方が{\jstrue} な場合、
 \begin{enumerate}
  \item current.\jsconfigurable が{\jsfalse}な場合、
  \begin{enumerate}
   \item current.\jswritable が{\jsfalse}でDesc.\jswritable が{\jstrue} な場合、Rejectします
   \item current.\jswritable が{\jsfalse}な場合、
   Desc.\jsvalue が存在し、SameValue(Desc.\jsvalue , current.\jsvalue)が{\jsfalse}な場合、Rejectします
  \end{enumerate}
  \item そうでなくて、current.\jsconfigurable が{\jstrue} な場合、
  変更が許可されます
 \end{enumerate}
 \item そうでなくて、IsAccessorDescriptor(current)とIsAccessorDescriptor(Desc)が双方とも{\jstrue} 
 な場合、
 \begin{enumerate}
  \item current.\jsconfigurable が{\jsfalse}な場合、
  \begin{enumerate}
   \item Desc.\jsset が存在し、SameValue(Desc.\jsset , current.\jsset)が{\jsfalse}な場合、Rejectします
   \item Desc.\jsget が存在し、SameValue(Desc.\jsget , current.\jsget)が{\jsfalse}な場合、Rejectします
  \end{enumerate}
 \end{enumerate}
 \item Oのプロパティ名Pのプロパティ属性をDescのもので書き換えます
 \item {\jstrue} を返します
\end{enumerate}

\section{ES5で追加されたオブジェクト関連のユーティリティメソッド}

ES3ではオブジェクトのプロパティの属性を変更したり、内部プロパティの
状態を変化させたりすることはできませんでした。
しかし、ES5では、より低レベルな操作ができるようなユーティリティメソッドが
定義されています。ここではそれらのメソッドについて
ES5仕様書のような厳密な解説はせずに
サンプルコードを交えた簡単な解説だけ行います。

\subsection{Object.getOwnPropertyDescriptor(O, P)}

\texttt{Object.getOwnPropertyDescriptor}メソッドは
プロパティディスクリプタをオブジェクト形式で取得します。

\begin{lstlisting}
var o = { foo: 1 };
var desc = Object.getOwnPropertyDescriptor(o, 'foo');
/*
プロパティディスクリプタdescは以下のようになります。
{
  value: 1,
  writable: true,
  enumerable: true,
  configurable: true
}
*/
\end{lstlisting}


\subsection{Object.defineProperty(O, P, Attributes)}

\texttt{Object.defineProperty}メソッドは
属性を指定してプロパティを定義したり、
すでにあるプロパティの属性を変更します。

\begin{lstlisting}
var o = {};
Object.defineProperty(o, 'foo', {
  value: 1,
  writable: false,
  enumerable: true,
  configurable: true
});

// fooプロパティが定義された
console.log(o.foo); // 1

// [[Writable]]がfalseなので読み出し専用となる
o.foo = 2;
console.log(o.foo); // 1

// [[Writable]]を更新
Object.defineProperty(o, 'foo', { writable: true });

// fooプロパティは書き込み可能となる
o.foo = 2;
console.log(o.foo); // 2
\end{lstlisting}

\subsection{Object.defineProperties(O, Properties)}

\texttt{Object.defineProperties}メソッドは
複数のプロパティに対して属性を指定してプロパティを定義したり、
すでにあるプロパティの属性を変更します。

\begin{lstlisting}
var o = {};
Object.defineProperties(o, {
  foo: {
    value: 1,
    writable: true,
    enumerable: true,
    configurable: true
  },
  bar: {
    get: function() { return this.foo },
    set: function(foo) { this.foo = foo }
  }
});
\end{lstlisting}


\subsection{Object.getPrototypeOf(O)}

\texttt{Object.getPrototypeOf}メソッドは
オブジェクトの内部プロパティ\jsproto を取得します。

\begin{lstlisting}
var proto = {};
console.log(Object.getPrototypeOf(proto) === Object.prototype); // true
var o = Object.create(proto);
consoel.log(Object.getPrototypeOf(o) === proto); // true
\end{lstlisting}


\subsection{Object.prototype.isPrototypeOf(V)}

\texttt{Object.prototype.isPrototypeOf}メソッドは
\jsthis が引数のオブジェクトの
プロトタイプチェーン上の\jsproto であるか判定します。

\begin{lstlisting}
var proto = {};
var o = Object.create(proto);
console.log(Object.prototype.isPrototypeOf(o)); // true
console.log(proto.isPrototypeOf(o)); // true

// Object.prototypeを継承していない場合は直接呼び出す
var proto = Object.create(null);
var o = Object.create(proto);
console.log(Object.prototype.isPrototypeOf.call(proto, o)); // true
\end{lstlisting}

\subsection{Object.create(O [, Properties])}

\texttt{Object.create}メソッドは
\jsproto を指定して新しいオブジェクトを作ります。

\begin{lstlisting}
// 同じ意味
var o = {};
var o = Object.create(Object.prototype);

// 第2引数からプロパティを設定できます(以下、同義)
var o = { foo: 1 };
var o = Object.create(Object.prototype, {
  foo: {
    value: 1,
    writable: true,
    enumerable: true,
    configurable: true
  }
});
var o = Object.create(Object.prototype);
Object.defineProperties(o, {
  foo: {
    value: 1,
    writable: true,
    enumerable: true,
    configurable: true
  }
});
\end{lstlisting}

\subsection{Object.isExtensible(O)}

\texttt{Object.isExtensible}メソッドはオブジェクトの
内部プロパティ\jsextensible を取得します。

\begin{lstlisting}
var o = {};
console.log(Object.isExtensible(o)); // true
\end{lstlisting}

\subsection{Object.preventExtensions(O)}

\texttt{Object.preventExtensions}メソッドはオブジェクトの内部プロパティ\jsextensible
を{\jsfalse}にします。つまり、プロパティを追加を禁止します。

\begin{lstlisting}
var o = {};
Object.preventExtensions(o);
console.log(Object.isExtensible(o)); // false

// プロパティの追加ができない
o.foo = 1;
console.log(o.hasOwnProperty('foo')); // false
\end{lstlisting}

\subsection{Object.seal(O)}

\texttt{Object.seal}メソッドはオブジェクトが持つ全ての
プロパティの\jsconfigurable 属性を{\jsfalse}にして、
オブジェクトの内部プロパティ\jsextensible を{\jsfalse}にします。
つまり、プロパティの追加、削除、属性の変更を禁止します。

\begin{lstlisting}
var o = { foo: 1 };
Object.seal(o);

// プロパティの追加ができない
o.bar = 1;
console.log(o.hasOwnProperty('bar')); // false

// プロパティの削除ができない
console.log(delete o.foo); // false

// プロパティ属性の変更ができない
console.log(Object.defineProperty(o, 'foo', {
  writable: false
})); // TypeError例外を投げる
\end{lstlisting}

\subsection{Object.isSealed(O)}

\texttt{Object.isSealed}メソッドは
全てのプロパティの\jsconfigurable 属性が{\jsfalse}で
オブジェクトの内部プロパティ\jsextensible が{\jsfalse}であるかどうかを
判定します。

\begin{lstlisting}
// Object.seal関数を使う
var o = { foo: 1 };
Object.seal(o);
console.log(Object.isSealed(o)); // true

// 手動
var o = Object.create(Object.prototype, {
  foo: {
    value: 1,
    writable: true
  }
});
Object.preventExtensions(o);
console.log(Object.isSealed(o)); // true
\end{lstlisting}

\subsection{Object.freeze(O)}

\texttt{Object.freeze}メソッドは
全てのデータプロパティの\jswritable 属性を{\jsfalse}、
全てのプロパティの\jsconfigurable 属性を{\jsfalse}にして、
内部プロパティ\jsextensible を{\jsfalse}にします。
つまり、全てのプロパティを読み出し専用にし、
プロパティの追加、削除、属性の変更を禁止します。

\begin{lstlisting}
var o = { foo: 1 };
Object.freeze(o);

// プロパティの追加ができない
o.bar = 1;
console.log(o.hasOwnProperty('bar')); // false

// プロパティの削除ができない
console.log(delete o.foo); // false

// プロパティ属性の変更ができない
console.log(Object.defineProperty(o, 'foo', {
  writable: false
})); // TypeError例外を投げる

// プロパティのが読み出し専用となる
o.foo = 2;
console.log(o.foo); // 1
\end{lstlisting}

\subsection{Object.isFrozen(O)}

全てのデータプロパティの\jswritable 属性が{\jsfalse}、
全てのプロパティの\jsconfigurable 属性が{\jsfalse}で
内部プロパティ\jsextensible が{\jsfalse}であるかどうかを判定します。

\begin{lstlisting}
var o = { foo: 1 };

// Object.freezeを使う
Object.freeze(o);
console.log(Object.isFrozen(o)); // true

// 手動
var o = Object.create(Object.prototype, {
  foo: { value: 1 }
});
Object.preventExtensions(o);
console.log(Object.isFrozen(o)); // true
\end{lstlisting}


\subsection{Object.keys(O)}

\texttt{Object.keys}メソッドは
オブジェクト自身が持つ\jsenumerable 属性がtrueである
プロパティのプロパティ名を配列で返します。

\begin{lstlisting}
var o = {
  foo: 1,
  bar: 2,
  baz: 3
};

console.log(Object.keys(o)); // ['foo', 'bar', 'baz'] (順番は保証されない)

// [[Enumerable]]属性がfalseなプロパティは無視する
Object.defineProperty(o, 'foo', { enumerable: false });
console.log(Object.keys(o)); // ['bar', 'baz']
\end{lstlisting}

\subsection{Object.getOwnPropertyNames(O)}

\texttt{Object.getOwnPropertyNames}メソッドはオブジェクト自身が持つ
全てのプロパティ(内部プロパティは含まない)名を配列で返します。

\begin{lstlisting}
var o = Object.create(Object.prototype, {
  foo: { value: 1 },
  bar: { get: function() { return 2 } },
  baz: { value: 3 }
});

console.log(Object.keys(o)); // []
console.log(Object.getOwnPropertyNames(o)); // ['foo', 'bar', 'baz']
\end{lstlisting}